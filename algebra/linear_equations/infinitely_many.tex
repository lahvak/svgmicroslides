\documentclass[dvisvgm]{standalone}
\usepackage{tikz}
\usepackage{simplesvgpres}
\usetikzlibrary{matrix, fit, decorations.pathreplacing, positioning, calc}

% Linear equation where all numbers are solutions

\begin{document}
\begin{tikzpicture}[]
    \begin{scope}[svgcls=layer1,
        node distance=0pt,
        every node/.style={inner xsep=0pt}]
        \matrix (line-1) [matrix of math nodes] {4 & (2x + 3) & {} - x & {}={}
            & 7x + 12\\};
    \end{scope}
    \begin{scope}[
        node distance=0pt,
        every node/.style={inner xsep=0pt}]
        \matrix (line-2) [matrix of math nodes, below=1ex of line-1-1-4.south,
        anchor=line-2-1-2.north] 
        {|[svgcls=layer3]| 8x + 12 - x &|[svgcls=layer5]| {}={}
            &|[svgcls=layer5]| 7x + 12\\};
    \end{scope}
    \begin{scope}[svgcls=layer7,
        node distance=0pt,
        every node/.style={inner xsep=0pt}]
        \matrix (line-3) [matrix of math nodes, below=1ex of line-2-1-2.south,
        anchor=line-3-1-2.north] 
        { 7x + 12 & {}={} & 7x + 12\\};
    \end{scope}
    \begin{scope}[svgcls=layer9,
        node distance=0pt,
        every node/.style={inner xsep=0pt}]
        \matrix (line-4) [matrix of math nodes, below=1ex of line-3-1-2.south,
        anchor=line-4-1-2.north] 
        { |[red]| -12 & \hphantom{{}={}} & |[red]| \hphantom{7x}{} - 12\\};
    \end{scope}
    \begin{scope}[svgcls=layer11,
        node distance=0pt,
        every node/.style={inner xsep=0pt}]
        \matrix (line-5) [matrix of math nodes, below=1ex of line-4-1-2.south,
        anchor=line-5-1-2.north] 
        { 7x & {}={} & 7x\\};
    \end{scope}
    \begin{scope}[svgcls=layer13,
        node distance=0pt,
        every node/.style={inner xsep=0pt}]
        \matrix (line-6) [matrix of math nodes, below=1ex of line-5-1-2.south,
        anchor=line-6-1-2.north] 
        {|[red]| -7x & \hphantom{{}={}} & |[red]| -7x\\};
    \end{scope}
    \begin{scope}[svgcls=layer15,
        node distance=0pt,
        every node/.style={inner xsep=0pt}]
        \matrix (line-7) [matrix of math nodes, below=1ex of line-6-1-2.south,
        anchor=line-7-1-2.north] 
        { 0 & {}={} & 0\\};
    \end{scope}
    \node[fit=(line-1) (line-2) (line-3) (line-4) (line-5) (line-6)] (bbox) {};
    \begin{scope}[svgcls=layer2, decoration=brace]
        \draw[thick, red] (line-1-1-1.south west) rectangle (line-1-1-1.north east);
        \draw[red, decorate] ($(line-1-1-2.north west)+(.2em,-.7ex)$) --
        ($(line-1-1-2.north east)-(.2em,.7ex)$);
        \draw[red, ->] (line-1-1-1.north) to [out=50, in=130] (line-1-1-2.north);
        \node[below=of bbox, align=left] {First distribute the $4$ on the left side.};
    \end{scope}
    \begin{scope}[svgcls=layer4]
        \node[below=of bbox, align=left] {The right side is already simplified
            as much as possible.};
    \end{scope}
    \begin{scope}[svgcls=layer6]
        \draw[thick, red] (line-2-1-3.south west) rectangle (line-2-1-3.north east);
        \node[below=of bbox, align=left] {Combine the like terms on the left:\\
            $8x - x = 7x$.};
    \end{scope}
    \begin{scope}[svgcls=layer8a]
        \node[below=of bbox, align=left] {At this moment we have the exact same
            expression\\on each side. No matter what $x$ is, these two sides
            \\will always be the same!};
    \end{scope}
    \begin{scope}[svgcls=layer8b]
        \node[below=of bbox, align=left] {For any possible value of $x$, the
            equation\\is true.  Any $x$ is a solution!};
    \end{scope}
    \begin{scope}[svgcls=layer8c]
        \node[below=of bbox, align=left] {We can do few more formal steps,
            though.};
    \end{scope}
    \begin{scope}[svgcls=layer8d]
        \node[below=of bbox, align=left] {Subtract $12$ from both sides.};
    \end{scope}
    \begin{scope}[svgcls=layer10]
        \node[below=of bbox, align=left] {Subtract $7x$ from both sides.};
    \end{scope}
    \begin{scope}[svgcls=layer12]
        \node[below=of bbox, align=left] {The equation becomes $0=0$.  That is
            a sure indicator\\ that we have infinitely many solutions.};
    \end{scope}
    \definePages{
        {layer1},
        {layer1, layer2},
        {layer1, layer2, layer3},
        {layer1, layer3, layer4, layer5},
        {layer1, layer3, layer5, layer6},
        {layer1, layer3, layer5, layer6, layer7},
        {layer1, layer3, layer5, layer7, layer8a},
        {layer1, layer3, layer5, layer7, layer8b},
        {layer1, layer3, layer5, layer7, layer8c},
        {layer1, layer3, layer5, layer7, layer8d, layer9},
        {layer1, layer3, layer5, layer7, layer9, layer11},
        {layer1, layer3, layer5, layer7, layer9, layer11, layer10, layer13},
        {layer1, layer3, layer5, layer7, layer9, layer11, layer13, layer12, layer15}}
\end{tikzpicture}
\end{document}
